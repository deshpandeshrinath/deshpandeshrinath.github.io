\documentclass[10pt]{article}

\title{Practical Mechanism Synthesis For Motion, Function and Path Generation Problem}
\date{}
\begin{document}
\maketitle
\section{Introduction}

Kinematic Synthesis of Mechanisms is a branch of Mechanical Design that deals with computing type and dimensions of machines with a certain desired application.
Depending upon the nature of application, mechanism synthesis is divided into three main categories: 1) Motion, 2) Path and 3) Function generation.
In Motion Generation, the goal is to navigate a rigid body through a prescribed positions and orientations.
Path generation aims to navigate a point through prescribed positions.
In the case of function generation, a prescribed relationship between the orientations of two ground links has to be maintained.
Considering the nature of approach, mechanism synthesis can also be categorized into two types: 1) Type, and 2) Dimensions Synthesis.

Four decades out literature of mechanism synthesis mainly

%## Optimal Synthesis for Extended Burmester Problem
%IDETC 17 Paper
%
%## Algorithm for Defect Free Linkage Synthesis using Database search for signatures
%IDETC 18 Paper
%
%## The AI Assistant
%Kinematic Synthesis of Mechanisms is an active research topic since past forty
%years.
%This has resulted in a wide range of computational and analytic methods for
%synthesizing various problem in kinematic synthesis of mechanisms.
%In spite of such a wealthy literature, the methods still have the following practical issues
%
%1. Given the highly nonlinear nature of the problem, the methods are highly sensitive to the input.
%   Thus they require very careful problem setup.
%2. The methods involve solving complex algebra or high dimensional optimization, which makes them **black boxes** from the designer's perspective. This hinders the designer in setting up the problem in a way that captures the designer's intent.
%
%In addition to that, there is a need for an interactive framework that lets the
%designer make the most of the synthesis methods in order to generate maximum
%feasible design concepts.
%
%In this proposal, I would like to present a framework which interacts with the user in a cohesive manner.
%An AI assistant is at the heart of this framework, which **<u>bridges the intuitive gap between the designer and computational solvers</u>**.
%1. The framework consists of an interface where the user can set up the problem.
%2. The AI assistant provides the user with suggestions to modify the input to maximize the outcome.
%3. The AI assistant then modifies and imputes the setup to suit the needs of computational solvers without sacrificing important aspects of the problem setup.
%4. The AI assistant communicates with appropriate solvers and processes the results.
%5. The AI assistant groups similar solutions together and provides the user with the
%   concept solutions along with important statistical information.
%   - e.g. Information such as robustness of mechanism
%
%The designer has the higher level control on the process of input modification and imputation. Think it as the designer deciding higher level control aspects such as where to go and the AI assistant has low-level control such as steering and accelerating.
%This kind of assistance removes the burden of mechanism designer by taking care
%of numerical technicalities, free choices; which are important for computational
%solvers but much less significant to the designer.
%<p align="center">
%<img src="{{ "/assets/proposal/ai-agent.jpeg" | absolute_url }}" alt="AI-Synthesis" style="width:40em">
%</p>
%An Example of Designer-AI Assistant Interaction would be as follows,
%
%- Mechanism Designer inputs some points as a guiding path. The objective is to generate linkages that follow a path and satisfy the Grashof condition.
%
%- For this case, a variational auto-encoder (VAE) is trained on coupler paths and motions  sampled from random grashof linkages. A recurrent neural network is trained to predict the orientations of coupler path points. The AI assistant in this case is comprised of these two networks.
%
%<p align="center">
%User Input: Guiding Points
%</p>
%<p align="center">
%<img src="{{ "/assets/proposal/ex1/user-input.png" | absolute_url }}" alt="AI-Synthesis" style="width:20em">
%</p>
%
%- An interpolating B-spline is fitted through these guiding points.
%
%$$X = Bspline(Guiding points) $$
%
%- In this case, user requires grashof linkages. Hence, the input should be modified such that
%   likelihood of getting grashof linkages should be maximum.
%   Thus, The B-spline is subjected to recognize the probability distribution of latent features corresponding to grashof trajectories.
%   Here, Latent Features $$z$$ are the hidden parameters of the generative model. The generative model (VAE) is trained to generate all possible type of closed loop coupler trajectories for four-bar, six-bar and slider-crank linkages.
%
%$$P(z|X) = Recognizer(X) $$
%
%$$\hat{X} = Generator(z)$$
%
%$$z_{sampled} \sim P(z|X)$$
%
%- Next, latent features are sampled from the recognized probability distribution for the B-spline fitted through guiding points as shown below.
%
%- Each point $$z = (z_1,z_2,z_3,..z_d)$$ in d dimensional latent space is represents features of an grashof trajectory.
%   Trajectories corresponding to the sampled features $$z_{sampled}$$ are generated using generator network. In this case, latent space is chosed to be 2-dimensional. As each sample is drawn from probability distribution $$ P(z|X) $$, trajectories generated by these samples will differ by a small amount. Also, User can control the region of sampling by interacting with latent space.
%
%$$\hat{X}_{samples} = Generator(z_{sampled})$$
%
%<p align="center">
%<img src="{{ "/assets/proposal/ex1/latent-features.png" | absolute_url }}" alt="AI-Synthesis" style="width:20em">
%
%<img src="{{ "/assets/proposal/ex1/sampled-paths.png" | absolute_url }}" alt="AI-Synthesis" style="width:20em">
%</p>
%
%
%- The AI assistant can provide the preferred types of the mechanisms (four-bar, slider-crank, six-bar) which are likely to satisfy this type of coupler paths. Thus, appropriate synthesis algorithms can be used as solvers.
%
%$$ P(Mechanism Type|z_{sampled}) = Type Classifier(z_{sampled})$$
%
%- Then AI assistant generates samples similar to the one provided by the user, which captures the main aspects of the path as shown above.
%
%- Let us assume that we can only use motion synthesis solver, which requires
%  orientation information along with the path.
%
%- Recurrent Neural Network trained on grashof coupler paths is used to attach orientation information (an example of Input Imputation).
%
%$$ \hat{\theta} = OrientationPredictor(\hat{X})$$
%
%- These samples are fed to computational solvers to yield solutions.
%
%$$ Concept Solutions = MotionSynthesis([\hat{X}, \hat{\theta}])$$
%
%Among the solutions obtained, following are the solutions which differ in the construction significantly. Hence, they can be termed as different feasible concepts for the application.
%
%<p align="center">
%<img src="{{ "/assets/proposal/ex1/sol1.svg" | absolute_url }}" alt="AI-Synthesis" style="height:15em">
%<img src="{{ "/assets/proposal/ex1/sol2.svg" | absolute_url }}" alt="AI-Synthesis" style="height:15em">
%<img src="{{ "/assets/proposal/ex1/sol3.svg" | absolute_url }}" alt="AI-Synthesis" style="height:15em">
%<img src="{{ "/assets/proposal/ex1/sol4.svg" | absolute_url }}" alt="AI-Synthesis" style="height:15em">
%<img src="{{ "/assets/proposal/ex1/sol5.svg" | absolute_url }}" alt="AI-Synthesis" style="height:15em">
%<img src="{{ "/assets/proposal/ex1/sol6.svg" | absolute_url }}" alt="AI-Synthesis" style="height:15em">
%<img src="{{ "/assets/proposal/ex1/sol7.svg" | absolute_url }}" alt="AI-Synthesis" style="height:15em">
%<img src="{{ "/assets/proposal/ex1/sol8.svg" | absolute_url }}" alt="AI-Synthesis" style="height:15em">
%<img src="{{ "/assets/proposal/ex1/sol9.svg" | absolute_url }}" alt="AI-Synthesis" style="height:15em">
%<img src="{{ "/assets/proposal/ex1/sol10.svg" | absolute_url }}" alt="AI-Synthesis" style="height:15em">
%<img src="{{ "/assets/proposal/ex1/sol11.svg" | absolute_url }}" alt="AI-Synthesis" style="height:15em">
%<img src="{{ "/assets/proposal/ex1/sol12.svg" | absolute_url }}" alt="AI-Synthesis" style="height:15em">
%</p>
%
%The AI assistant groups the solutions according to the similarity of the concepts. These solution groups are presented to mechanism designer.
%
%The above example is implemented to show the proof of concept.
%However, the approach is not limited to the above example.
%
%The AI assistant is proposed to be trained for doing the following things:
%
%### Interactive Input Suggestion
%- Suggest the next action for the user in the process of setting up the problem.
%  - e.g. Suggesting locations/ regions for fixed pivots, suggesting next
%    precision pose
%
%### Imputation and Modification of Input
%- Generating path/motion trajectories that are close to input and are more likely to have a good
%  solution.
%- Generating variational samples such that a particular quality is preserved.
%  - e.g. Path walking mechanisms, it is important that the path contains a flat portion (flat part of the D shaped curve). Thus all the variations of the input should have a flat part.
%- Generating Conditional variations in the input to produce the desired
%  conditions on output linkages. (e.g. generate variational inputs such that all pivots are more likely to lie above a line.)
%- Changing line/path/region constraints within a tolerance, such that it is more likely that they lead to good solutions.
%- Providing Missing Orientations
%- Auto-filling the free choices in Six-bar motion generation to produces good
%  defect free six-bar linkages.
%- Predicting timing information for path synthesis
%
%### Type Synthesis
%- Probability distribution assignment for a set of mechanism types based on the problem
%  setup.
%  - e.g. For a given guided points, six-bar linkage is more likely to perform
%    the task than fourR or slidercrank linkage.
%
%### Processing the solutions
%This problem becomes more apparent when solver returns a large number of solutions and also for linkages with many links.
%- Group similar solutions together to form a concept group
%- Give a rank for each concept solution based on its fitness for the given task, and present the solutions/concept groups in order of the ranks.
%- Asses the robustness of each solution
%
%### Another Path Generation Example
%<p align="center">
%User Input &nbsp; --> &nbsp; Feature Detection &nbsp; --> &nbsp;Input Sampling
%</p>
%<p align="center">
%<img src="{{ "/assets/proposal/ex2/user-input.png" | absolute_url }}" alt="AI-Synthesis" style="width:15em">
%<img src="{{ "/assets/proposal/ex2/feature-samples.png" | absolute_url }}" alt="AI-Synthesis" style="width:15em">
%<img src="{{ "/assets/proposal/ex2/samples-paths.png" | absolute_url }}" alt="AI-Synthesis" style="width:15em">
%</p>
%
%<p align="center">
%Mechanism Concepts
%</p>
%<p align="center">
%<img src="{{ "/assets/proposal/ex2/sol1.svg" | absolute_url }}" alt="AI-Synthesis" style="height:15em">
%<img src="{{ "/assets/proposal/ex2/sol2.svg" | absolute_url }}" alt="AI-Synthesis" style="height:15em">
%<img src="{{ "/assets/proposal/ex2/sol3.svg" | absolute_url }}" alt="AI-Synthesis" style="height:15em">
%</p>
%
%<p align="center">
% Overview of All Solutions
%</p>
%<p align="center">
%<img src="{{ "/assets/proposal/ex2/all-solutions.png" | absolute_url }}" alt="AI-Synthesis" style="width:100%">
%</p>
%
%The above image depicts all of the solutions found for the input reconstructed
%from 60 features samples drawn from recognized probability distribution.
%The image indicates the distribution of fixed (black circles) and moving pivots (red circles) for feasible linkages for the given task.

\end{document}
